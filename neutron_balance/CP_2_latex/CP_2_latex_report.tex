\documentclass[11pt, oneside]{article}   	% use "amsart" instead of "article" for AMSLaTeX format
\usepackage{geometry}                		% See geometry.pdf to learn the layout options. There are lots.
\geometry{letterpaper}                   		% ... or a4paper or a5paper or ... 
%\geometry{landscape}                		% Activate for rotated page geometry
\usepackage[parfill]{parskip}    		        % Activate to begin paragraphs with an empty line rather than an indent
\usepackage{graphicx}				% Use pdf, png, jpg, or eps§ with pdflatex; use eps in DVI mode
\usepackage{float}				        % TeX will automatically convert eps --> pdf in pdflatex		
\usepackage{amssymb}
\usepackage{listings}
\usepackage{amsmath}
%SetFonts

%SetFonts


\title{CP 2 - Neutron Balance \\
\ NPRE 247}
\author{Nalin Gadihoke}
\date{03/30/2018}							% Activate to display a given date or no date

\begin{document}
\maketitle
\thispagestyle{empty}
%\section{}
%\subsection{}


\clearpage


\tableofcontents{}
\thispagestyle{empty}
\clearpage
\pagenumbering{arabic}
Project was written in Latex and programming was done in Python.
\section{Theory}

For this project, a program was written to read cross section data, set up Migration and Fission matrices which were then used to setup the equation $B\phi = k\phi $. The Eigenvectors and values were then calculated for the neutron balance problem. 

\begin{table}[h!]
  \begin{center}
    \caption{List of Variables}
    \label{tab:table1}
    \begin{tabular}{l|r} 
      \textbf{Symbol} & \textbf{Variable}\\
      \hline
      L & loss of neutrons through Leakage\\
      A & loss of neutrons through absorption\\
      $S_{o}$ & scattering out of neutrons\\
      $S_i$ & scattering in of neutrons\\
      F & neutrons produced in fission\\
      $\chi$&fission spectrum\\
      $\phi$&neutron flux\\
      $\nu$&neutrons per fission\\
      $\sum _{a}$&absorption cross section\\
      $\sum _{s,g\leftarrow g'}$&scattering cross section from g' to g\\
      g & energy group\\
      G&total number of groups\\
      k & eigenvalue\\
    \end{tabular}
  \end{center}
\end{table}
 
\subsection{Neutron Balance in Infinite Space}

Neutron balance refers to the equilibrium in the gain of neutrons and loss of neutrons. This is shown by the following equation:
\begin{equation}
L+A+S_{o} = F + S_{i}
\end{equation}

Since the reaction rate is equal to the cross-section multiplied by the flux and since in infinite space we will consider no neutron diffusion taking place, $L=0$ and (1) can be written in terms of $\sum$ and $\phi$ as

\begin{equation}
(\sum _{a} + \sum _{s,1\rightarrow g})\phi  = (\sum _{s,g\rightarrow 1} + \frac{1}{k}\chi \nu \sum _{f})\phi 
\end{equation}

For a multi-group problem we evaluate these variables as matrices with the same naming convention as before (a matrix is denoted with lines above the variable name corresponding to dimensions). Hence, (2) takes the form of  

$$\bar{\bar{A}}\bar{\phi } + \bar{\bar{S}}_{out}\bar{\phi } - \bar{\bar{S}}_{in}\bar{\phi } - \frac{1}{k}\bar{\bar{f}}\bar{\phi } = 0$$

rearranging, we get
\begin{equation}
k(\bar{\bar{A}}+ \bar{\bar{S}}_{out} - \bar{\bar{S}}_{in})\bar{\phi } - \bar{\bar{f}}\bar{\phi } = 0\end{equation}

Let the migration matrix be \begin{equation}\bar{\bar{M}} = \bar{\bar{A}}+ \bar{\bar{S}}_{out} - \bar{\bar{S}}_{in}\end{equation}Therefore, 
\\
$$k\bar{\phi } = \bar{\bar{M}}^{-1}\bar{\bar{f}}\bar{\phi }$$
Let matrix B be 
$$\bar{\bar{B}} = \bar{\bar{M}}^{-1}\bar{\bar{f}}$$
Therefore, we have our final equation:
\begin{equation}k\bar{\phi } = \bar{\bar{B}}\bar{\phi }\end{equation}

\subsection{Multi-group Migration and Fission Matrices}
Now we define our three matrices according to inputs (provided in input file) and (3)
\begin{align*}
\bar{\bar{A}}= \begin{bmatrix}
 \sum _{a1} & 0 & 0\\ 
 0 & \sum _{ai} &  0 \\ 
 0  & 0 &  \sum _{aG}
\end{bmatrix}
\end{align*}

$$\bar{\bar{S}}_{out}  = \begin{bmatrix}
\sum_{i}^{G}(\sum_{s,g=1\rightarrow i} )  & 0 & 0\\ 
 0& \dots &0 \\ 
0 & 0 & \sum_{i}^{G}(\sum_{s,g=G\rightarrow i} )
\end{bmatrix} $$

$$\bar{\bar{S}}_{in}  = \begin{bmatrix}
0&\sum_{s,g=i\rightarrow 1} & \sum_{s,g=G\rightarrow 1} \\ 
 \dots& \dots & \dots \\ 
\sum_{s,g=1\rightarrow G}  & \sum_{s,g=i\rightarrow G}  & 0 
\end{bmatrix} $$

$$\bar{\bar{f}}  = \begin{bmatrix}
\chi_{i} \nu_{i} \sum_{fi}&\dots & \chi_{i} \nu_{i} \sum_{fG} \\ 
 \dots& \dots & \dots \\ 
\chi_{G} \nu_{G} \sum_{fi}  & \dots & \chi_{G} \nu_{G} \sum_{fG} 
\end{bmatrix} $$

$$\bar{\phi }  = \begin{bmatrix}
\phi _{g=1}  \\ 
 \dots \\ 
\phi _{g=G} 
\end{bmatrix} $$

The migration matrix is constructed according to (4)
$$\bar{\bar{M}} = \begin{bmatrix}
 \sum _{a1} & 0 & 0\\ 
 0 & \sum _{ai} &  0 \\ 
 0  & 0 &  \sum _{aG}
\end{bmatrix} + \begin{bmatrix}
0&\sum_{s,g=i\rightarrow 1} & \sum_{s,g=G\rightarrow 1} \\ 
 \dots& \dots & \dots \\ 
\sum_{s,g=1\rightarrow G}  & \sum_{s,g=i\rightarrow G}  & 0 
\end{bmatrix} -  \begin{bmatrix}
\chi_{i} \nu_{i} \sum_{fi}&\dots & \chi_{i} \nu_{i} \sum_{fG} \\ 
 \dots& \dots & \dots \\ 
\chi_{G} \nu_{G} \sum_{fi}  & \dots & \chi_{G} \nu_{G} \sum_{fG} 
\end{bmatrix}$$

\newpage
\section{2G Eigenvalue/Eigenvector Problem}
\subsection{Inputs}
Inputs were provided in a table format and can be found in the input file in the programming part of the project.
\subsection{The Matrices}
All necessary matrices are made according to outlines in section 1.2
$$
\bar{\bar{A}} = \begin{bmatrix}
0.0092 &0 \\ 
0 & 0.0932
\end{bmatrix} \bar{\bar{S}}_{out} = \begin{bmatrix}
0.0202 &0 \\ 
0 & 0
\end{bmatrix} \bar{\bar{S}}_{in} = \begin{bmatrix}
0 &0 \\ 
0.0202 & 0
\end{bmatrix}$$
\\
$$ \bar{\bar{M}} = \begin{bmatrix}
0.0294 &0 \\ 
-0.0202 & 0.0932
\end{bmatrix}\bar{\bar{f}} = \begin{bmatrix}
0.0046 &0.1139 \\ 
0 & 0
\end{bmatrix}\bar{\bar{B}} = \bar{\bar{M}}^{-1}\bar{\bar{f}} = \begin{bmatrix}
0.1564&3.8741 \\ 
0.0339 & 0.8396
\end{bmatrix}$$

\subsection{Analytical Solution}

Assuming $I_{G}$ is an identity matrix, we can rewrite (5)
$$(k\bar{\bar{I_{G}}} - \bar{\bar{B}})\bar{\phi } = 0$$
From the previous subsection, we have matrix B
$$\bar{\bar{B}} =  \begin{bmatrix}
0.1564&3.8741 \\ 
0.0339 & 0.8396
\end{bmatrix}$$
Therefore, we deduce that the equation will have a solution only when 
$$ determinant(k\bar{\bar{I_{G}}} - \bar{\bar{B}}) = (k-0.1564)(k-0.8396) + (3.8741*0.0339) = 0 $$
This simplifies to the following, 
$$k^{2} - 0.99613k - 1.110*10^{-16}= 0$$
giving us roots $k_1 = 0.99613$ and $k_2 = 0$. These can be solved independently; for the first root the following steps were followed
$$( \begin{bmatrix}
 0.99613&0 \\ 
0 &  0.99613
\end{bmatrix}-\begin{bmatrix}
0.1564&3.8741 \\ 
0.0339 & 0.8396
\end{bmatrix} ) \bar{\phi } = \begin{bmatrix}
0.8396&-3.8741 \\ 
-0.0339 & 0.1564
\end{bmatrix} \bar{\phi } = 0$$

$$\begin{bmatrix}
0.8396&-3.8741 \\ 
0.8057 & -3.7177
\end{bmatrix} \begin{bmatrix}
\phi _{1'}\\ 
\phi _{2'}
\end{bmatrix}= \begin{bmatrix}
0\\ 
0
\end{bmatrix}$$

$$\phi _{1'}(0.8057+0.8396)-\phi _{2'}(3.8741+3.7177) = 0$$

Therefore we get
\begin{equation}
\phi _{1'}=4.6142\phi_{2'}, \bar{\phi _{1}} = \begin{bmatrix}
-0.9773\\ 
-0.2118
\end{bmatrix}(\frac{neutrons}{cm^{2}s}), k_{1} = 0.99613 \end{equation}

And $k_2$ is solved as follows
$$( \begin{bmatrix}
 0&0 \\ 
0 &  0
\end{bmatrix}-\begin{bmatrix}
0.1564&3.8741 \\ 
0.0339 & 0.8396
\end{bmatrix} ) \bar{\phi } = \begin{bmatrix}
-0.15646&-3.8741 \\ 
-0.0339 & -0.8396
\end{bmatrix} \bar{\phi } = 0$$

$$\begin{bmatrix}
-0.1564&-3.8741 \\ 
-0.1903 & -4.7137
\end{bmatrix} \begin{bmatrix}
\phi _{1'}\\ 
\phi _{2'}
\end{bmatrix}= \begin{bmatrix}
0\\ 
0
\end{bmatrix}$$

$$\phi _{1'}(0.1564+0.1903)+\phi _{2'}(3.8741+4.7137) = 0$$

\begin{equation}
\phi _{1'}=-24.77\phi_{2'}, \bar{\phi _{2}} = \begin{bmatrix}
-0.9991\\ 
0.04035
\end{bmatrix}(\frac{neutrons}{cm^{2}s}), k_{2} = 0 \end{equation}

\subsection{Solution using the python 'eig' function}

To make use of a computer to verify the results, the 'eig' function in python was used, which accepts an array and returns the Eigen vectors as column vectors and Eigen values as a single vector. These were the data structures computed

$$vectors = \begin{bmatrix}
 -0.99918547& -0.97730867\\ 
 0.04035341&-0.21182012 
\end{bmatrix}, values = \begin{bmatrix}
 0&0 \\ 
 0& 0.996138799
\end{bmatrix}$$

Written out separately, the match the values analytically calculated in (7) and (8)

$$\phi _{1'}=4.6142\phi_{2'}, \bar{\phi _{1}} = \begin{bmatrix}
-0.9773\\ 
-0.2118
\end{bmatrix}(\frac{neutrons}{cm^{2}s}), k_{1} = 0.99613 $$
$$
\phi _{1'}=-24.77\phi_{2'}, \bar{\phi _{2}} = \begin{bmatrix}
-0.9991\\ 
0.04035
\end{bmatrix}(\frac{neutrons}{cm^{2}s}), k_{2} = 0 
$$

\subsection{Power Iteration}
Next, we used power iteration on the Eigen values and vectors with given initial starting values as 1 for both the vector and value. The largest of both were calculated using the formula given to us
\begin{equation}\bar{\phi }_{i+1} = \frac{\bar{\bar{B}}\bar{\phi }_{i}}{\left \| \bar{\bar{B}}\bar{\phi }_{i} \right \|}_{2}\\
k_{i+1} = \frac{(\bar{\bar{B}}\bar{\phi }_{i})^{T}\bar{\phi }_{i+1}}{\bar{\phi }_{i+1}^{T}\bar{\phi }_{i+1}}\end{equation}
This was implemented via a for loop and the maximum values were found to be as follows in the second iteration
$$\bar{\phi} _{i=2} = \begin{bmatrix}
0.97730867\\ 
0.21182012
\end{bmatrix}(\frac{neutrons}{cm^{2}s}), k_{i=2} = 0.9961388
$$
\newpage
\section{8G Eigenvalue/Eigenvector Problem}
\subsection{Inputs}
Inputs were provided in a table format and can be found in the input file in the programming part of the project. 
\subsection{The Matrices}

In this section, the matrices became too complicated to do by hand so functions were implemented to populate the arrays accordingly. Here are the migration and fission matrices
$$\bar{\bar{M}} = \begin{bmatrix}
0.0888 & 0 & 0 & 0 & 0 & 0 & 0 &0 \\ 
 -0.053& 0.1193 &0  & 0 & 0 & 0 & 0 &0 \\ 
 -0.0301 &-0.1159 & 0.0813    & 0 &0  &0  &0  &0 \\ 
 -0.0001& -0.0005 & -0.0769 & 0.2152 & 0 & 0 & 0 & 0\\ 
 0     & 0     &-0.0019 &-0.1961  &0.2529 & 0    &  0     & 0  \\ 
 0    & 0    & 0    &-0.005  &-0.1737  &0.3437 &-0.0023 & 0    \\ 
 0     &0     & 0     &-0.0007 &-0.0246 &-0.2707 & 0.417 & -0.0275\\ 
 0&0  & 0 & -0.0001 & -0.0073 & -0.055 & -0.3589 & 0.2073
\end{bmatrix}$$

$$\bar{\bar{f}} = \begin{bmatrix}
 0.004699 & 0.001963 & 0.0003857 & 0.002349 & 0.007715  & 0.007785
 &  0.03145 & 0.07508\\ 
 0.005500 &  0.002298  & 0.0004515 & 0.002750 & 0.009031   & 0.009113
  & 0.03682 & 0.08788 \\ 
 0.003199 & 0.001337 & 0.0002626 & 0.001599 & 0.005253  & 0.005301
  & 0.02142 & 0.05112 \\ 
 0&0  &0  &0  &0  & 0 & 0 &0 \\ 
0&0  &0  &0  &0  & 0 & 0 &0\\ 
 0&0  &0  &0  &0  & 0 & 0 &0\\ 
 0&0  &0  &0  &0  & 0 & 0 &0 \\ 
 0&0  &0  &0  &0  & 0 & 0 &0
\end{bmatrix}$$

\subsection{Solution using the python 'eig' function}
Section 2.4 goes over how the following matrices were calculated. The matrix vectors holds the Eigen Vectors in each of its columns
\begin{align*}
vectors = \begin{bmatrix}
-0.9827& -0.2615   & -0.3405&
  -0.2160 &-0.2160 & 0.6949
  &-0.3218&-0.3218 \\ 
0.06596&-0.3441&-0.6733&0.7237&  0.7237&  -0.04546 &-0.1006& -0.1006\\ 
 0.1499&-0.7819&-0.6305&-0.3516&-0.3516&-0.3168&-0.6704& -0.6704 \\ 
 0.05374&-0.28035& -0.03253&0.24800& 0.24800&0.6409&0.2277& 0.2277\\ 
0.04279&-0.2232&-0.1093&-0.1374&-0.1374& 0.0287& 0.01155& 0.01155 \\ 
0.02254& -0.1176&-0.07239&-0.04970&-0.04970&0.04220&-0.05812&-0.05812 \\ 
 0.02003&-0.1045&0.1207& 0.02256& 0.02256&-0.02836& 0.01663& 0.01663 \\ 
 0.042200& -0.2201    &  0.01135& -0.001538&-0.001538&-0.01606&0.01695& 0.01695
\end{bmatrix}
\end{align*}

$$values = \begin{bmatrix}
0&1.090031 \\ 
 0&0 \\ 
 0&0 \\ 
 0&0 
\end{bmatrix}$$

\subsection{Power Iteration}
Equation (8) was used here too and the following vector and value were obtained after the second iteration. The output file contains values from the second iteration onwards as described in the problem. 
$$\bar{\phi} _{i=2} = \begin{bmatrix}
0.26158725\\ 
0.3441239\\0.78197886
 \\ 0.280355  
 \\ 0.22326364
 \\ 0.1176114 
 \\ 0.10450765
 \\ 0.22013644
\end{bmatrix}(\frac{neutrons}{cm^{2}s}), k_{i=2} = 1.090031
$$

\section{Results}
This computer project gave us a good insight into eigenvalues and how to solve for them in a neutron diffusion problem. 

For the 2G case, analytical values calculated by hand matched those produced by the 'eig' function which matched the largest values calculated using the power iteration method demonstrating different ways to go about solving a similar problem. 
For the 8G case, the power iteration method converged on the value calculated by the 'eig' function further demonstrating the method's validity.  
\section{References}

1. www.sharelatex.com
2. www.stackoverflow.com

\end{document}